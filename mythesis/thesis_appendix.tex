%------------------------------------------------------------------------------
\chapter{Appendix}
\label{sec:app}
%------------------------------------------------------------------------------

\section{Discrete Group Theory}

\section{Conventions}

\subsection{\texttt{Chroma} Euclidean Dirac Matrices}
$\gamma_0 = 
\begin{bmatrix}
    0 & 0 & -1 & 0 \\
    0 & 0 & 0 & -1 \\
    -1 & 0 & 0 & 0 \\
    0 & -1 & 0 & 0 \\
    \end{bmatrix}$
\\
$\gamma_1 = 
\begin{bmatrix}
    0 & 0 & 0 & -i \\
    0 & 0 & -i & 0 \\
    0 & i & 0 & 0 \\
    i & 0 & 0 & 0 \\
    \end{bmatrix}$
\\
$\gamma_2 = 
\begin{bmatrix}
    0 & 0 & 0 & -1 \\
    0 & 0 & 1 & 0 \\
    0 & 1 & 0 & 0 \\
    -1 & 0 & 0 & 0 \\
    \end{bmatrix}$
\\
$\gamma_3 = 
\begin{bmatrix}
    0 & 0 & -i & 0 \\
    0 & 0 & 0 & i \\
    i & 0 & 0 & 0 \\
    0 & -i & 0 & 0 \\
    \end{bmatrix}$
\\
$\gamma_4 = -\gamma_0$ \\
$\gamma_5 = \gamma_0\gamma_1\gamma_2\gamma_3 = 
\begin{bmatrix}
    1 & 0 & 0 & 0 \\
    0 & 1 & 0 & 0 \\
    0 & 0 & -1 & 0 \\
    0 & 0 & 0 & -1 \\
    \end{bmatrix}$
\\
$C = -\gamma_0\gamma_2$

\texttt{Chroma} uses the following conventions \cite{Edwards_2005} For particles of spin-1 can {\em arbitrarily } define a sign convention:
\begin{align}
  \rho(k) &\equiv \bar{\psi} \gamma_k \psi \nonumber \\
  \varrho(k) &\equiv \bar{\psi} \gamma_k \gamma_4 \psi \nonumber\\
  a_1(k) &\equiv \bar{\psi} \gamma_k \gamma_5 \psi \nonumber\\
  b_1(k) &\equiv \bar{\psi}\tfrac{1}{2}  \epsilon_{ijk} \gamma_i \gamma_j \psi \nonumber
\end{align}
With this convention some of the chroma gamma matrices carry a minus sign when creating a state.\\

\vspace{1cm}

\begin{tabular}{c|c|c|c| r| c}
$n_\Gamma(\mathrm{dec})$ & $n_\Gamma(\mathrm{bin})$ & name & $\Gamma$ & state & $\widetilde{n_\Gamma}(\mathrm{dec})$\\
\hline
0 & 0000 & a0 & $1$ & $a_0$ & 15\\
1 & 0001 & rho\_x & $\gamma_1$ & $\rho(x)$ & 14\\
2 & 0010 & rho\_y & $\gamma_2$ & $\rho(y)$ & 13\\
3 & 0011 & b1\_z & $\gamma_1 \gamma_2$ & $b_1(z)$ & 12\\
4 & 0100 & rho\_z & $\gamma_3$ & $\rho(z)$ & 11\\
5 & 0101 & b1\_y & $\gamma_1 \gamma_3$ & $- b_1(y)$ & 10\\
6 & 0110 & b1\_x & $\gamma_2 \gamma_3$ & $b_1(x)$ & 9\\
7 & 0111 & pion\_2 & $\gamma_1 \gamma_2 \gamma_3 = \gamma_5 \gamma_4$ & $\pi$& 8 \\
8 & 1000 & b0 & $\gamma_4$ & $b_0$ & 7 \\
9 & 1001 & rho\_x\_2 & $\gamma_1 \gamma_4$ & $\varrho(x)$ & 6\\
10 & 1010 & rho\_y\_2 & $\gamma_2 \gamma_4$ & $\varrho(y)$ & 5\\
11 & 1011 & a1\_z & $\gamma_1 \gamma_2 \gamma_4 = \gamma_3 \gamma_5$ & $a_1(z)$ & 4\\
12 & 1100 & rho\_z\_2 & $\gamma_3 \gamma_4$ &  $\varrho(z)$ & 3\\
13 & 1101 & a1\_y & $\gamma_1 \gamma_3 \gamma_4 = - \gamma_2 
\gamma_5$ & $- a_1(y)$ & 2\\
14 & 1110 & a1\_x & $\gamma_2 \gamma_3 \gamma_4 = \gamma_1 \gamma_5$ & $a_1(x)$ & 1\\
15 & 1111 & pion & $\gamma_1 \gamma_2 \gamma_3 \gamma_4 = \gamma_5$ &  $\pi$ & 0 \\

\end{tabular}


