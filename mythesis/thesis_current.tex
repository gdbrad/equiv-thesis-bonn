% !TEX root = mythesis.tex

%==============================================================================
\chapter{Current Working Notes}\label{sec:current}
%==============================================================================

Here we describe current works in progress for ultimately determining the meson spectrum for all $J^{PC}$ values using LQCD. By using more statistics, several quark masses, and a range of lattice spacings, we can obtain precise continuum masses using local and non-local operators, as well as mesons at rest and in flight. 

In order to calculate the correlation between interpolating operators at time separation $t$ in the vacuuum, we must solve a GEVP of size \todo{TBD}. We can then calculate the masses of the transfer matrix eigenstates. 

Consider a quark at $x_1$ and an antiquark at $x_2$, joined by a gauge-invariant product of links along a path $P$ 
\begin{equation}
    \bar{\psi}(x_2) \prod_P U \Gamma \psi(x_1)
\end{equation}
with $\Gamma$ belonging to \cref{Table:gamma} following Chroma conventions. The recipe for forming the class of interpolators given this construction is to take a gamma structure along with linear combinations of different paths defined by covariant derivative operators, thus allowing access to particular continuum quantum numbers. 



\section{Local operators}
Here there is no path $P$ and $x_1=x_2$. 
the possible gamma structures are 
\begin{tabular}{c|c|c}
    $\gamma$ & $J^{PC}$ & name \\
    1 & $0^++$ & $\delta$ \\
    $\gamma_4$ & $0^+-$ & exotic \\
    $\gamma_i$ & $1^--$ & $\rho$ \\
    $\gamma_i \gamma_4$ & $1^--$ & $\rho$ \\
    $\gamma_5$ & $0^-+$ & $\pi$ \\
    $\gamma_5 \gamma_4$ & $0^-+$ & $\pi_2$ \\
    $\gamma_i \gamma_j$ & $1^+-$ & B \\
    $\gamma_5 \gamma_i$ & $1^++$ & $A_1$
\end{tabular}

We must form a correspondence between the transformation properties of the path P and combine them with quark bilinears. Thus, we must classify the path $P$ under symmetries of the lattice, namely, rotation, reflection, translation, and charge conjugation. the process of restricting SU(2) representations to the allowed cubic rotations and classifying them under the octahedral group $O_h$ is called \textit{subduction}. 

Lets walk through the required input to perform contractions involving different types of operators:

\subsection{operator list}
\begin{verbatim}
    operator_list = g5.g3.d(0,0,1).p0:g5.g2.d(0,1,0).p0
    operator_list = g3.d0.p0,1,2:g6.d0.p0,1,2
    operator_list = g5.d0.p0,1,2
    operator_list = g4.d0.p0,1,2    
\end{verbatim}


