% !TEX root = mythesis.tex

%==============================================================================
\chapter{Introduction}
\label{sec:intro}
\newcommand{\todo}[1]{\textbf{\color{red}TODO: #1}}

%==============================================================================
\todo{put in summary from motivation letters}
Quantum Chromodynamics is the SU(3) gauge field theory governing the strong force, dictating the dynamics of quarks and gluons. One can have two or three quark states, mesons and baryons, respectively, bound together by gluons. In continuum QCD, the regime in which the coupling constant of the theory is still small, perturbation theory can reliably be used. However, to probe the hadronic world ($\mu \leq 1 GeV$) , perturbation theory ceases to be a useful tool. By discretizing Minkowski spacetime into Euclidean spacetime, we can calculate the hadron spectrum and other quantities such as mattrix elements.
This non-perturbative approach is called Lattice QCD(henceforth LQCD), the only systematically improvable and regularization scheme independent method to probe the strong force. It is crucial to note that this formulation is \textbf{not} a model; LQCD preserves 
Hadronic states that do not fit into the traditional quark model have been coined XYZ states \todo{cite{xyz review article}} or exotics. Exotic hadrons are interesting for a host of reasons: 

\todo{cite} http://arxiv.org/abs/1709.01417

The class of exotics that we aim to explore are doubly charmed tetraquarks in isospin channels $I=0,1$. Namely, the tetraquark with flavor content $cc\bar{u}\bar{d}$ known as $T_{cc}^+(3875)$. The flavor content is is based on the decay channel $D^0D^0\pi^+$ and has isospin 0. Apparently, \textbf{check references for this}, this is the longest-lived exotic hadron. This exotic has mass of roughly 3875 MeV and manifests as a peak in the mass spectrum of $D^0D^0\pi^+$ mesons.  We will use meson-meson interpolators to explore these isospin quantum numbers as opposed to diquark-antidiquark operators; The use of the latter was previously the gold standard in the study of exotic hadrons, but as of late, dimeson operators are primarily used when the heavy quark mass is \textbf{not} close to the bottom quark mass. For instance, it is posited that $T_{bb}$ is likely a diquark-antidiquark state, thus, employing diquark-antidiquark operators is important in this case.  

The end game of this study is to \textbf{establish a pole in the corresponding scattering amplitude $t(E_{cm})$}. Moreover, we will investigate whether this species of tetraquark exists within the $DD^*$ threshold. We can extract the scattering amplitude from a Luscher analysis of the lattice data, thereby obtaining the finite volume energies. 

Phenomenologically, we will study if attraction occurs near the $DD*$ threshold for the former isospin quantum number and repulsion for the latter. We will heed  Christoph's expertise at this stage of the analysis. Recently, [[what years]]the LHCb collaboration has discovered exotic hadrons. Very few [[which ones exactly]] have been studied on the lattice and thus lack a rigorous theoretical basis.
\section{Goal}
Construct color singlet tetraquark interpolating operators using the distillation framework with dimeson interpolating operators. Compute the spectrum and energy shifts with respect to the $DD^*$ threshold for a heavy quark mass close to the charm quark mass. Perform a Luscher analysis to obtain the scattering amplitude. We would like to confirm that the $DD^*$ interaction is repulsive in the $I=1$ channel and attractive in the $I=0$ channel, which logically follows from the $I=0$ assignment for the $T_{cc}^+$ state. 

We aim to investigate tthe construction of exotic color-singlet combinations eg. $T_{cc}$ with flavor $cc\bar{u}\bar{d}$ by calculating the $DD^*$ scattering amplitude with LQCD. In Chapter 1 we will elucidate the spectroscopy methods that we employ on the lattice, namely quark field smearing with distillation, operator construction, and two point correlators. In chapter 2 we describe the computational workflow, tools involved, and the HPC cost associated with the calculation of the eigenbasis, meson elementals, perambulators. In chapter 3 we lay out in detail interpolating operator construction for meson and di-mesons within the distillation framework, derivative (extended) operators, and how to account for mesons at non-zero momentum. The relevant group theory is introduced, to be expanded on in the appendix. In chapter 4 we expand on the end-point analysis of meson correlators and present results for our study of mesonic signal saturation with distillation, specifically the dependence on the size of the distillation basis and number of source insertions. In chapter 5 we describe remaining work, notably a rigourous lattice determination of the quark mass dependence of the $T_cc(3875)$ using a large basis of $DD^*$ interpolating operators in various irreducible reprsentations and total momenta. A brief exposotion of the Luescher method for locating poles from the scattering amplitude is provided. 

1. First compute quark propagators for each gauge configuration
2. These are then combined to construct hadron propagators 
3. Average over all gauge configurations -> estimate of hadron propagators 
4. Perform end-point analysis to extract meaningful physical observables


