% !TEX root = mythesis.tex

%==============================================================================
\chapter{Introduction}
\label{sec:intro}
\newcommand{\todo}[1]{\textbf{\color{red}TODO: #1}}

%==============================================================================
Quantum Chromodynamics is the SU(3) gauge field theory governing the strong force, dictating the dynamics of quarks and gluons. One can have two or three quark states, mesons and baryons, respectively, bound together by gluons. QCD phenomena are informed by two main quark correlations: confinement, whereby color forces allow $q,\bar{q}$ to be correlated into color singlets, and chiral symmetry breaking. In continuum QCD, the regime in which the coupling constant of the theory $\alpha_s$ is still small, perturbation theory can reliably be used. However, to probe the hadronic world ($\mu \leq 1 GeV$), perturbation theory ceases to be a useful tool. By discretizing Minkowski spacetime into Euclidean spacetime, we can calculate the hadron spectrum and other quantities such as matrix elements.  This non-perturbative approach is called Lattice QCD(henceforth LQCD), the only systematically improvable and regularization scheme independent method to probe the strong force. It is crucial to note that this formulation is \textbf{not} a model; The property of quark confinement can be realized in the strong coupling approximation of the lattice version of QCD. For more information on the underlying theory, consult \cite{10.5555/3029317},\cite{gupta1998introductionlatticeqcd},\cite{Gattringer2009QuantumCO},\cite{Griffiths:1987tj},\cite{Cheng1984GaugeTO}.


Hadronic states that do not fit into the traditional quark model have been coined XYZ states \cite{Brambilla:2019esw} or exotics, such as tetraquark and pentaquark states\cite{Cheung_2017}. Multiquark states are posited to be combinations of conventional mesons or diquark-antidiquark pairs, which imply colored building blocks. QCD should predict whether tetraquark states exist, thus, we can leverage the lattice to construct color-flavor-spatial-spin structures resembling that of compact tetraquarks, such as the $T_{cc}^+(3875)$, which is the focus of this project.  The lack of a consensus as to how the experimental data is to be interpreted, namely the make-up of the aforementioned multiquark building blocks, has spawned a flurry of research efforts in the Lattice QCD community~\cite{Cheung_2017}. The inner workings of QCD and the nature of unstable hadronic resonances, which comprise most of the observed spectrum to date,  will be further illuminated once the grand question of interpretation is resolved. Very few exotic hadrons have been studied on the lattice and thus lack a rigorous theoretical basis.

The class of exotics that we aim to explore are doubly charmed tetraquarks in isospin channels $I=0,1$. Namely, the tetraquarks with flavor content $\bar c\bar s ud$, $c\bar s u\bar d$, $cc\bar u\bar d$ and $c\bar c u\bar d$; The third flavor profile is known as $T_{cc}^+(3875)$ \cite{LHCb:2021vvq}. The flavor content is based on the decay channel $D^0D^0\pi^+$ and has isospin 0. The experimental data shows that this is the longest-lived exotic hadron. This exotic has mass of roughly 3875 MeV and manifests as a peak in the mass spectrum of $D^0D^0\pi^+$ mesons.  We will use meson-meson interpolators to explore these isospin quantum numbers as opposed to diquark-antidiquark operators; The use of the latter was previously the gold standard in the study of exotic hadrons, but as of late, dimeson operators are primarily used when the heavy quark mass is \textbf{not} close to the bottom quark mass. For instance, it is posited that $T_{bb}$ is likely a diquark-antidiquark state, thus, employing diquark-antidiquark operators is important in this case.  

The aim of this study is to \textbf{establish a pole in the corresponding scattering amplitude $t(E_{cm})$} using distillation smearing on the lattice with coupled dimeson interpolating operators. Assuming we have a suitably large basis of interpolators in the relevant channels of interest, we can compute the spectrum and energy shifts with respect to the $DD^*$ threshold for a heavy quark mass close to the charm quark mass. Moreover, we will investigate whether this species of tetraquark exists within said threshold. We can extract the scattering amplitude from a L\"{u}scher analysis of the lattice data, thereby obtaining the finite volume energies. We endeavor to show that lattice calculations are in agreement with phenomenology, namely, that the $DD^*$ interaction is repulsive in the $I=1$ channel and attractive in the $I=0$ channel, which logically follows from the $I=0$ assignment for the $T_{cc}^+$ state. 


In Chapter 1 we describe contemporary hadron spectroscopy methods that we employ on the lattice,  namely quark field smearing with distillation, operator construction, and two point correlators. In chapter 2 we describe the computational workflow, tools involved, and the HPC cost associated with the calculation of the eigenbasis, meson elementals, perambulators. In chapter 3 we lay out in detail interpolating operator construction for meson and di-mesons within the distillation framework, derivative (extended) operators, and how to account for mesons at non-zero momentum. The relevant group theory is introduced, to be expanded on in the appendix. In chapter 4 we expand on the end-point analysis of meson correlators and present results for our study of mesonic signal saturation with distillation, specifically the dependence on the size of the distillation basis and number of source insertions. This study dictates the ideal rank of the distillation basis to use when calculating the perambulators and elementals for each ensemble; We can proceed to compute the spectrum and energy shifts with respect to the $DD^*$ threshold for a heavy quark mass close to the charm quark mass.  In chapter 5 we describe remaining work, notably a rigorous lattice determination of the quark mass dependence of the $T_{cc}(3875)$ using a large basis of $DD^*$ interpolating operators in various irreducible representations and total momenta. A brief exposition of the L\"{u}scher method to obtain finite volume energies from the scattering amplitude is provided. Finally, we must search for poles, which indicate where an attractive potential is not deep enough to hold a bound state, thus permitting us to make a phenomenological interpretation from the lattice analysis. 