% !TEX root = mythesis.tex

%==============================================================================
\chapter{Hadron Spectroscopy on the Lattice}
\label{sec:spec}
%==============================================================================
% \newcommand{\todo}[1]{\textbf{\color{red}TODO: #1}}
\section{Background}
We first provide a brief overview of continuum QCD and how to extract low-energy observables using lattice QCD. In the following section we describe the nature of the so-called signal-to-noise problem which plagues calculations in LQCD, most notably the extraction of hadronic observables. As the Euclidean time between operators increases, the signal-to-noise ratio degrades exponentially. It is standard practice to use quark field smearing algorithms to circumvent this problem and ultimately extract ground-state energies with better statistics. Next, we introduce the mature method of distillation \cite{peardon_novel_2009} and the improvements to the signal of correlation functions in contrast to traditional smearing techniques. This will spell out the theoretical basis for the proceeding chapter on operator construction and the computational implementation of described methods to ultimately capture the mesonic and di-meson spectrum, setting the stage for exotic hadrons. 

\section{Continuum QCD}
QCD is the gauge theory of strong interactions with color $SU(3)$ as the underlying gauge group. The color degree of freedom was introduced into the quark model to satisfy the pauli princicple. The matter-fields in the theory are quarks (spin-$\frac{1}{2}$) fermions with six flavors and three possible colors (we don't detect these in experiment!). No quark has ever been observed as an asympotically-free state. During the 1970's, Gell-Mann\cite{Gell-Mann:1964ewy}and others \cite{PhysRevD.8.3633}\cite{PhysRevLett.31.494} tackled the question of exactly which symetry of the quark model should be gauged. The ``hidden'' color degree of freedom possessed by quarks is plausible since we only observe color-singlets in nature. Thus, the strong forces between quarks with color need be colorless. Therefore, We can write down the QCD Lagrangian from the free-quark Lagrangian by applying the gauge principle with respect to the color gauge group $SU(3)$. The fundamental objects are the quark fields  $q_{\alpha,A}^{(f)}$ where $\alpha: 1,\dots,4$ are the Dirac-spinor indices, $f: 1,\dots,6$ denotes the flavor, and  $A: 1,2,3$ specifies the \text{color} index in the fundamental representation. 
The QCD Lagrangian is 
\begin{align}
    \mathcal{L}_{QCD} = \sum_{f=1}^{6} \sum_{a,b=1}^{3} \sum_{\alpha,\beta,\mu=0}^{3} \bar{q}_{\alpha a}^{(f)}(i(\gamma^\mu)_{\alpha \beta}(\mathcal{D}_{\mu;ab}) - m_f\delta_{\alpha \beta}\delta_{ab}) q_{\beta b}^{(f)} + \sum_{i=1}^{8}\sum_{\mu,\nu=0}^{3} -\frac{1}{4} G_{\mu\nu}^i G_i^{\mu\nu}
\end{align}
where $G_{\mu\nu} = \partial_\mu A_\nu - \partial_\nu A_\mu - ig[A_\mu,A_\nu]$ contains the gluons (gauge fields) $A_\mu = \sum_{a=1}^{8}A_\mu^a \lambda^a/2$ where $\lambda^a$ are the Gell-Mann matrices that index $1,\dots 8$ corresponding to the adjoint representation of $SU(3)$ . The covariant derivative $\mathcal D_\mu$ acts on the quark fields like so $\mathcal D_\mu q_k = (\partial_\mu - igA_\mu)q_k$ 
$\mathcal{L}_{QCD}$ exhibits discrete symmetries charge conjugation, parity \todo{expand}

\subsection{Path integral formulation of QCD}
Wilson \cite{Wilson:1974sk} first introduced lattice gauge theory in 1974 where both space and time are discretized. The quantization of the gauge fields is done via a Wick rotation to Euclidean space $t\rightarrow i\tau$. The integration shown in  

With modern computing architecture, we are able to study numberous observvables with the path-integral formalism coupled with the Monte Carlo random walk method  




Integrating over Minkowski spacetime, we obtain the QCD action 
\begin{align}
    S_{QCD} = \int d^4x \{\underbrace{-\frac{1}{4} \sum_{a=1}^{8}F_{\mu\nu}^a F^{a\mu\nu}}_{S_G[A]}+ \underbrace{\sum_{f}\sum_{i,j=1}^{3}\sum_{\alpha,\beta=1}^{4}\bar{q}_{i,a}^f(x)[i(\gamma^\mu)_{\alpha\beta}(\mathcal{D}_\mu)_{ij} - m_f\delta_{ij}\delta_{\alpha\beta}]q_{j,\beta}^f(x)}_{S_f[A,\bar{q},q]}\}
\end{align}
 

\section{Lattice QCD }

QCD is defined as LQCD as the lattice spacing goes to 0; With infinite compute time we can exactly solve QCD under one assumption, which is that we are in the right phase of the theory.  It needs to be emphasized that LQCD is \textbf{not} a model, rather, it is the only non-perturbative regularization scheme in the infared regime that is systematically improvable, meaning one can dial the input parameters. This improvement manifests itself via the following properties of discretized spacetime, which are tightly coupled to high-performance computing architecture. The crucial feature of this discretized version of the continuum theory is gauge covariance, namely $m_{gluon} \rightarrow 0$, which is the only way the theory can remain renormalizable without adding new terms. 
\begin{itemize}
    \item Discretization scale ($a$) where there are no infinities at finite values of $a$
    \item Finite Volume 
    \item Unphysical quark mass, which allows us to extrapolate to the physical quark mass with effective field theory 
\end{itemize}

Discretization of a continuum theory naturally lends itself well to being transcribed onto a computer. In essence, we perform LQCD calculations on some HPC cluster, encode known properties of QCD into some EFT (eg. Chiral perturbation theory, Baryon $\chi$PT, etc. which will not be discussed in this work) to make predictions about the standard model. The tunable parameters of LQCD simulations are the coupling constant $\alpha_s$ and the quark masses modulo the top quark.  Modern computing power allows us to perform these at the physical pion mass. We will walk through lattice operator construction to create a diqaurk $[\bar{q}q]$ of any spin ($J \in \mathbb{Z}$) and parity $P$ with two quark fields and an insertion of the covariant derivative $\nabla$ with gamma strucutre $\Gamma$, the product of which determines the spin $J$.  

\subsection{Discretization of the Fermion and Gauge Actions}


\subsection{Wilson Fermion Action}

\section{Gauge Field Smearing}
Gauge links, the edges on the lattice connecting the vertices(quark sites), are the discretized version of gluons in the continuum theory of QCD. They are known to introduce noise in the form of short range fluctuations. To circumvent this, different types of ``smearing'' are typically applied during the generation of ensembles. 

In order to achieve overlap with states of interest in the continuum, namely the low-lying states, one must being with smearing of the quark fields via some smoothing function. A brief exposition of \textit{Jacobi smearing} will follow. 
\begin{equation}
    \tilde{\psi}_{a\alpha}(x) = \mathcal{S}_{ab}(x,y) \psi_{b\alpha}(y)
\end{equation}
Here, $x,y$ are lattice sites, $a,b$ are color indices, $\alpha$ a spin component. 

This construction is derived from the ``parent'' representation of a gauge-invariant, spatially symmetric operation, which serves as a means of improving the projection onto low-lying states in correlation functions:
\begin{align}
    \tilde{\Psi}(\vec{x},t) = \sum_{\vec{y}} L(\vec{x},\vec{y})\psi(\vec{y},t)
\end{align}
Define the jacobi smearing operator as \todo{prob dont need this}
\begin{align}
\nabla
\end{align}
We employ 6-stout smearing \cite{stout} and add smear the quark fields using distillation in the proceeding phase of the computational chain. \cite{peardon_novel_2009}. Here we briefly describe stout smearing and arrive at the definition of a stout link. 

\subsection{Stout Smearing}
Each gauge link on the lattice has four neighboring staples. 
Let $C_\mu(x)$ denote the weighted sum of the perpendicular staples which
begin at lattice site $x$ and terminate at neighboring site 
$x\!+\!\hat{\mu}$, this formulation was introduced by \cite{stout}:

\begin{eqnarray}
 C_\mu(x)&=&\sum_{\nu\neq \mu}\rho_{\mu\nu}\biggl(
 U_\nu(x) U_\mu(x\!+\!\hat{\nu}) U_\nu^\dagger(x\!+\!\hat{\mu})\nonumber\\
&&+ U^\dagger_\nu(x\!-\!\hat{\nu}) U_\mu(x\!-\!\hat{\nu})
  U_\nu(x\!-\!\hat{\nu}\!+\!\hat{\mu})
\biggr), \label{eq:Cdef}
\end{eqnarray}

This is the formulation of a plaquette, where the magnitude of the vectors $\hat{u}, \hat{V}$ is the length of a single lattice spacing (given by the particular ensemble) and the weights $\rho_{\mu\nu}$ are tunable. 

\section{Distillation Smearing}
The computation of all-to-all quark propagators is crucial in LQCD calculations of correlation functions for a suitably large set of interpolators. This is computationally expensive as the number of ``solves'' of the lattice Dirac matrix grows \todo{how does it scale for all to all maybe romero paper}. By projecting quark fields into a subspace(with rank less than that of the original \todo{hilbert space/vector space?}), we can reliably compute all-to-all propagators\cite{peardon_novel_2009}. Within this reduced subspace, we can project operators onto definite momentum at both the source and sink. Hermiticity of operators is also guarunteed with this method, which bodes well for solving the GEVP within each irrep of interest. As scattering studies require well-controlled momentum insertions, this will serve us well at later stages of the $T_cc(3875)$ analysis.  

The information we need to extract is 
$$ V^{\dagger}M^{-1}V \rightarrow \tau $$ 
where $\tau$ is the perambulator matrix on a single time slice. 
$V(t)$ is a matrix with $4 \times N_v $  columns constructed from eigenvectors of the covariant 3d Laplace operator. It is important to note that $V(t)$ does not act on Dirac components. Thus, $V(t)$ is a block identity in Dirac space and each block contains the first $N_v$ eigenvectors $v_i(t)$. A given column $V^{(i,\alpha)}(t)$ has entries 
$$ V^{(i,\alpha)}(t)_{\vec{x},t',\beta} = v_i(t)_{\vec{x}} \delta_{tt'}\delta_{\alpha\beta}$$


Propagators transform with tensor product structure 
$$\text{Lattice} \otimes \text{Matrix(Nc)} \otimes \text{Matrix(Ns)} \otimes \text{Complex}$$
\textbf{check this and format with a figure from the TALK AT NRW FAIR }
We can work out these dimensions for ourselves; A distilled propagator stored on disk has dimensions 
$ 2 * 8 *2 *4 *4 *10 * 10 * 16$ 
with the dictionary 
$$\text{} \text{complex} * \text{snk} * \text{src} * N \times N_{\sigma} * \text{tslice}$$
\section{Contractions}
At this point, we need to actually perform contractions to obtain the correlator. This is a sum over all of the spin and color permabulator indices with spin matrices and color tensors, thus creating colorless objects satisfying the gauge invariance of QCD. We analyze the Monte Carlo ensembles of correlation functions on a configuration by configuration basis. 
\begin{align}
    \label{ops_cc}
    O=\sum_{i,j} A_{ij}  [(\bar u\Gamma_i c)( p^1_{i})(\bar d\Gamma_{j} c)(p^2_{i}) -  (\bar d \Gamma_i c)({ p^1_{i}})(\bar u\Gamma_{j} c)({ p^2_{i}}) ]\nonumber,
    \end{align}

    where $\Gamma_{i/j}$ represents the appropriate Dirac structure and the momenta $p^1$ and $p^2$ add up to zero, as we will stay in the rest frame within this period. 
$$C_M^{(2)}(t',t) = Tr[\Phi^B(t')\tau(t',t)\Phi^A(t)\tau(t,t')]$$ 
where 
$$\Phi^A_{\alpha\beta}(t) = V^{\dagger}(t) [\Gamma^A(t)]_{\alpha\beta} V(t) \equiv V^{\dagger}(t)\mathcal{D}^A(t)V(t)S^A_{\alpha\beta}$$ 
and 
$$\tau_{\alpha\beta}(t',t) = V^{\dagger}(t')M_{\alpha\beta}^{-1}(t',t)V(t)$$ 
is the perambulator, defined by the lattice representation of the Dirac operator, $M$. 

