% !TEX root = mythesis.tex
%==============================================================================
\chapter{Computational Considerations}\label{sec:comp}
%==============================================================================
The amount of compute time for generating the eigenbasis, and subsequently the perambulators and meson elementals, scales with the spatial extent of the lattice $N$ and the rank of the distillation basis $n$. We provide an overview of our lattice setup and a summary of the Hybrid Monte Carlo method for generating gauge configurations. Next, we present the software stack and pipeline for our calculations on the Jureca cluster, including the storage and runtime cost associated with each object necessary for performing hadron spectroscopy with distillation. As this framework hinges upon algebraic mulitigrid methods, we touch on the key points and how this technology interfaces with \texttt{Chroma} lattice objects. At the final stage, matrix elements(meson elementals) and perambulators are contracted together with some appropiate Dirac strucutre depending on the flavor and isospin channel (the $J^{PC}$ continuum quantum numbers to be accessed), which we elucidate diagramatically. The details of our 6-stout smeared ensembles are given for completion. Having a wide range of pion masses and lattice spacings to study will allow us to probe the quark mass dependence of the exotic doubly charmed tetraquark state $T_{cc}^+$.   

\section{Lattice Setup}
The only known \textit{ab-initio} approach to probing the low energy regime of QCD is with High Performance Computing(HPC) and the discretized lattice. At bottom, a path-integral approach is used to compute some observable $\mathcal{O}$ with a predefined collection of bare physics and lattice parameters$\{\beta,am_{ud},am_s\}$. Again, the lattice path integral is:
\begin{align}
  Z = \int [DU] \text{exp}\left[-S[U]\right] \quad \text{where} \quad  DU = \prod_{x,\mu}^{} dU_{\mu x}
\end{align} where one can add in the dependence on the bare physics and lattice parameters like so: 
\begin{equation}
  \bra{}\mathcal{O}\ket{} = \frac{1}{Z} \int \mathcal{D}[U] \mathcal{O}(U) \cdot p(U)
\end{equation} where $p(U)$ is built from the discretized Dirac operator and pure gauge action and the integral measure is $\mathcal{D}[U] = \prod_{x=0,\mu=1}^{d,V} dH(U_\mu(x))$ where $V$ is the volume of the lattice in $d=4$ dimensions (the lattice is a hypercube, a 4D object) and the Haar measure $dH$, which allows the lattice path integral to retain gauge invariance. The data in this problem spaces is contained in the lattice sites. 

In  we showed how the discretization of the euclidean path integral reduces the infinite number of integrals to something more manageable. However, the number of lattice sites on a typical lattice is $\approx32^4$, so a ``brute-force'' approach is not feasible. See~\cite{Montvay_Münster_1994}, \cite{Luscher:2010ae},\cite{Creutz:1988wv}~\cite{Finkenrath:2023sjg} for more details. To make the problem tractable, we turn to Markov Chain Monte Carlo simulations to calculate the expectation value and correlation functions of observables 
\begin{equation}
  \left<\mathcal{O}\right> = \frac{1}{N} \sum_{q,\bar{q},U}^{} \mathcal{O}\left[q,\bar{q},U\right] 
\end{equation} where $\left[q,\bar{q},U\right] \approx \text{exp}(-S[q,\bar{q},U])$. The latter is the probability weight that we use to perform importance sampling of the gauge field configurations. This is necessary because only a small vicinity of the minimum of the spectral density will contribute. Moreover, The Boltzmann factor $\text{exp}(-S[\phi])$ dictates the distribution of configurations in the sample produced by the Monte Carlo integration. This process is comprised of two steps \cite{Luscher:2010ae}: 
\begin{itemize}
  \item Proposal: a new gauge configuration $U'$ is generated with some weight $q(U)$ with probability $T(U \to U')$ 
  \item Correction or Updating: $P_{accept}(U,U') = \min \left[1,\frac{p(U')q(U)}{p(U)q(U')}\right]$
\end{itemize} 
Assuming that the above ratio follows a log-normal distribution then the acceptance rate is $\texttt{erfc}\left(\sqrt{\sigma^2/8}\right)$ where erfc is the complementary error function. The collection of configurations $U_i$ with associated weight $p(U)$ make up the ensemble. 


\section{Software Stack for Distillation}
Distillation is costly initially both in storage and component construction. This pays off at the end of the day as we can reuse the perambulators for subsequent calculations; The inversions can be precomputed and stored on disk. For the di-meson system we are investigating, the contraction cost is not the dominant contribution. We will use the MultiGrid (MG) solver from \texttt{QUDA}, \texttt{Chroma} with \texttt{Superbblas} support, the \texttt{PRIMME} eigensolver, and \texttt{Numpy Einsum} for contractions. The amount of computation and storage scales with the lattice size $N$ and the rank of the distillation basis, $n$. The optimal rank of the distillation basis is determined experimentally, but it is proportional to the spatial volume of the lattice. All of the high-peformance computing tasks below were performed on the Jureca cluster at the Juelich Supercomputing Centre. 
\section{Pipeline}
In order to perform spectroscopy calculations for a given ensemble, there exists a sequential dependency chain for HPC calculations:
\begin{itemize}
     \setlength\itemsep{1em}
        \item[\checkmark] \textbf{Ensemble generation:} $N_f = 2+1$ quark flavors, a tree level Symanzik improved gluon action and 6-stout dynamical smeared Wilson fermions
        \item[\checkmark] \textbf{HPC Tasks:} Generation of distillation basis, perambulators, meson elementals using \texttt{Chroma} with \texttt{superbblas} support on the Jureca cluster at JSC
        \item Construct \textbf{Di-meson distilled operators} using Hadspec method of subduction coefficents and helicity operators
        \item[\checkmark] Perform \textbf{contractions} of multi-hadron operators $\to$ 2pt correlators
        \item Construct correlation function coming from the \textbf{GEVP} in the right irreducible representation
    \end{itemize}
\begin{figure}[!htbp]
    \begin{center}
      \scalebox{1.0}{
        \begin{tikzpicture}[node distance=1.5cm and 3cm] % Adjusted node distance
            \tikzstyle{startstop} = [rectangle, rounded corners, minimum width=3cm, minimum height=1cm, text centered, draw=black, fill=red!30]
            \tikzstyle{io} = [trapezium, trapezium left angle=70, trapezium right angle=110, minimum width=3cm, minimum height=1cm, text centered, draw=black, fill=blue!30]
            \tikzstyle{process} = [rectangle, minimum width=3.5cm, minimum height=1cm, text centered, text width=3cm, draw=black, fill=orange!30]
            \tikzstyle{arrow} = [thick,->,>=stealth]

            % Nodes
            \node (start) [startstop] {Input File Generation: XML and SLURM Batch Scripts};
            \node (in1) [io, below of=start] {Calculate Distillation Basis/Eigenpairs (GPU)};
            
            % Parallel processes
            \node (pro1) [process, below of= in1, xshift=-3cm, yshift=-1cm] {Meson Elementals};
            \node (pro2) [process, below of= in1, xshift=3cm, yshift=-1cm] {Perambulators};
            
            % H5 Conversion processes
            \node (h5conv1) [process, below of=pro1] {H5 Conversion};
            \node (h5conv2) [process, below of=pro2] {H5 Conversion};
            
            % Contracting
            \node (contract) [process, below of=h5conv1, xshift=3cm] {Contract Elementals and Perambulators};
            
            % Output
            \node (out1) [io, below of=contract] {Output the Two-Point Correlator};
            \node (stop) [startstop, below of=out1] {H5 output ($N_t \times N_{cfg}$)};

            % Arrows
            \draw [arrow] (start) -- (in1);
            \draw [arrow] (in1) -- (pro1);
            \draw [arrow] (in1) -- (pro2);
            \draw [arrow] (pro1) -- (h5conv1);
            \draw [arrow] (pro2) -- (h5conv2);
            \draw [arrow] (h5conv1) -- (contract);
            \draw [arrow] (h5conv2) -- (contract);
            \draw [arrow] (contract) -- (out1);
            \draw [arrow] (out1) -- (stop);
        \end{tikzpicture}
      }
    \end{center}
    \caption{Flowchart of the computational workflow for computing lattice objects with distillation.}
    \label{fig:alg_flow}
\end{figure}
\section{Multigrid Methods in LQCD}
Computing fundamental building blocks on the lattice, which will each be discussed in the proceeding sections, requires sophisticated hardware architecure. We use the \verb|Chroma| software stack from USQCD \cite{Edwards_2005} compiled with \verb|superbblas| support (required for computing the distilled objects of interest), and the multigrid solver from \verb|QUDA|. We run the HPC tasks on a single GPU node, as we found multi-node jobs are not compatible with our stack and the Jureca cluster at JSC. This stack ultimately allows us to calculate a large class of matrix elements at high precision \cite{romero_efficient_2020} using distillation. 

Before we dive into the details of the calculation of our distilled objects, it is useful to show how lattice objects in \texttt{Chroma} transform under different spaces with a tensor product structre \cite{Edwards_2005}. 
\begin{table}
  \centering
\begin{displaymath}
  \begin{array}{llcccccl}
                    & {\textit Lattice} &    &  {\textit Color} &     & {\textit Spin}  &     & {\textit Complexity}\\
  {Gauge\ fields:}& {\texttt Lattice} &\otimes& {\texttt Matrix(Nc)}&\otimes& {\texttt Scalar}    &\otimes& {\texttt Complex}\\
  {Fermions:}    & {\texttt Lattice} &\otimes& {\texttt Vector(Nc)}&\otimes& {\texttt Vector(Ns)}&\otimes& {\texttt Complex}\\
  {Scalars:}     & {\texttt Scalar}  &\otimes& {\texttt Scalar}    &\otimes& {\texttt Scalar}    &\otimes& {\texttt Scalar} \\
  {Propagators:} & {\texttt Lattice} &\otimes& {\texttt Matrix(Nc)}&\otimes& {\texttt Matrix(Ns)}&\otimes& {\texttt Complex}\\
  {Gamma:}       & {\texttt Scalar}  &\otimes& {\texttt Scalar}    &\otimes& {\texttt Matrix(Ns)}&\otimes& {\texttt Complex}\\
  \end{array}
  \end{displaymath}
  \caption{The tensor structure of \texttt{Chroma} objects. \texttt{Nd}: num. space-time dimensions, \texttt{Nc}: dimension of the color vector space, \texttt{Ns}: dimension of the spin vector space.
  \cite{Edwards_2005}}
    \label{fig:chroma}
\end{table}
Gauge fields can left-multiply fermions via color matrix $\times$ color vector but is diagonal in spin space (spin scalar $\times$ spin vector).
A $\Gamma$ matrix can right-multiply a propagator (spin matrix $\times$ spin matrix) but is diagonal in color space (color matrix $\times$ color
scalar). Say we have two lattice fermion fields $A, B$, a lattice color matrix, or gauge field \texttt{U}. At each lattice site, \texttt{U} is a scalar in spin space. The product 
\begin{equation}
  B_\alpha^i (x) = U^{ij}(x) \times A_\alpha^j(x)
\end{equation} for lattice coordinates $x$, $i,j$ are the color indices, and $\alpha$ is the spin index. The tensor multiiplication is over complex types, from which we will extract the real component at the stage of forming two point hadron correlators.  

  
\subsection{Eigenpairs/Distillation Basis Calculation}
Sparse linear algebra is necessary to construct the smeared quark fields based on distillation, where the
eigenvectors with smallest eigenvalues of the gauge-covariant Laplace operator $\nabla^2[t]$. 
This is the first step in the calculation which is input data for the generation of elementals and perambulators. We use the \verb|Primme| eigensolver to obtain the eigenpairs(eigenvalues,eigenvectors) of the Hermitian Wilson-Dirac operator\cite{PRIMME}\cite{Frommer:2020ovr}. 
The eigenbasis in distillation space contains the eigenvectors from the lowest-lying regime of the spectrum of the hermitian negative-definite Laplacian-like operator 
\begin{equation}
  \nabla^{2(\vec{x},\vec{y})} = 6\delta_{\vec{x},\vec{y}} - \sum_{\vec{j}\in|(1,0,0),(0,10),(0,0,1)|}^{} U_{\vec{j}}^{(x)}\delta_{\vec{x}+\vec{j},\vec{y}} + U_{\vec{j}}^{(\vec{x}-\vec{j})^\dagger}\delta_{\vec{x}-\vec{j},\vec{y}}
\end{equation}
where the gauge fields $U$ are stout smeared. 
Distillation smearing projects into the $N_D$ dimensional vector space like so 
\begin{equation}
  \Box_{ab}(x_1,x_2,t) = V(x_1,t)^{d}_a V^{\dagger} (x_2,t)^{d}_b
\end{equation} where $a,b$ are the color indices and $d = 1 \dots N_D$ is the distillation space index~\cite{10.5555/3029317}. This is the eigenbasis in distillation space, which now allows us to access elements of the quark propagator.  


\subsection{Calculating Meson Elementals}
The calculation of our elemental building blocks requires tensor contractions. These matrix-matrix multiplications are accelerated with the \texttt{Superbblas} library \cite{dinapoli2013efficientuseblaslibrary}. These objects are calculated for every time-slice, on every gauge configuration, and with every combination of displacements. If we disregard the perambulators for a moment, the correlation functions are described by the correlation between meson elementals at the different time-slices. We can write the distilled meson elemental as 
\begin{equation}
\Phi^A_{\alpha\beta}(t) = V^{\dagger}(t) [\Gamma^A(t)]_{\alpha\beta} V(t) \equiv V^{\dagger}(t)\mathcal{D}^A(t)V(t)S^A_{\alpha\beta}
\end{equation} where $\mathcal{D}^A(t)$ is the lattice Dirac matrix and $S^A_{\alpha\beta}$ is the subduction coefficient, which is the identity for $\vec{P^2} = 0$.
For non-zero displacement, the $\Gamma$ operator is no longer trivial in position and color space and requires single or double covariant derivatives, $\nabla_i$ where $i$ is in the x,y,or z direction, to act on the 3D spinor components at some time $t$. In the later section on derivative operators, we will show how this is expressed in the language of $\texttt{Chroma}$. For completeness, the elemental form for zero, single, and double derivative displacement, respectively, are: 
\begin{eqnarray}
  & V^{\dagger}(t) [\Gamma^A(t)]_{\alpha\beta} V(t) \\ 
  & V^{\dagger}(t) [\nabla_k(t)]_{\alpha\beta} V(t) \\ 
  & V^{\dagger}(t) [\nabla_k(t)\nabla_l(t)]_{\alpha\beta} V(t)
\end{eqnarray}
This tensor contraction involves the following\cite{romero_efficient_2020}: 
\begin{itemize}
  \item Two 5-dimensional tensors (the distillation basis) $v,w$, the coordinates of which are on the \textbf{spatial} extent of the lattice $N_L$, three color space components, and the distillation basis column index 
  \item  4-dimensional tensor $z$ representing a phase $z^{\vec{x},l} = e^{-i\vec{p}\vec{x}}$ which carries a component in $N_L$ as well as a momentum index
  \item color contraction with the projector into color space, the Kronecker delta, where $\vec{x}\in N_t,\alpha,\beta \in C$
  \subitem $ M^{i,j,l} = \sum_{\vec{x},\alpha,\beta} \delta^{\alpha\beta} v^{\vec{x},\alpha,i}w^{\vec{x},\beta,j} z^{\vec{x},l} \quad \text{for} \quad 1\leq i,j \leq n, 1\leq l \leq M$ 
\end{itemize}
which is performed at different time slices in parallel. 

\subsection{Calculating Perambulators}
The perambulator characterizes the distilled version of a quark propagator from time $t_0$ to some later time $t_{\text{f}}$ and all combinations therein for the entire temporal extent $N_t$ of the lattice \cite{peardon_novel_2009}
\begin{equation}
  \tau[t_0,t_f]_{ij}^{\alpha\beta} = v_{i,\alpha}[t_0]^{\dagger} D^{-1}v_{j,\beta}[t_f]
\end{equation} where we can factorize these $4N_v \times 4N_v$ matrices like so 
\begin{equation}
  (4 \times 4) \otimes (N_v \times N_v)
\end{equation} yielding $16N_v^2$ inner products and $4N_v$ inversions. One can immediately form the perambulator $\tau(t_0,t_f)$ for all values of $t_0$ after solving for the vector $D^{-1}v_{j\beta}[t_f]$ on each configuration; Each configuration will have its own perambulator binary file containing information over all possible time slices. We use the Wilson-Clover multigrid solver from $\texttt{QUDA}$ \cite{quda} specifically the GCR solver(a Krylov solver) to calculate the linear system of equations 
\begin{equation}
  Dx^{i\alpha} = v_{i\alpha}(t)
\end{equation} where $i$ is the rank of the distillation basis, $\alpha$ is a color index, and $t$ is indexed along the temporal extent of the lattice. 
\section{Cost and Storage of Distillation}
      As we highlighted in the previous section, distillation restricts operators to a small subspace while maintaining overlap with relevant eigenstates. As the computational cost is directly proportional to the size of matrices one is manipulating, a reduction in the rank of operators slashes the cost of computing the propagation matrix. Once a suitable set of perambulators are computed, we can reuse them to correlate a collection of interpolators.
        \begin{table}[H]
          \centering
        \begin{minipage}{\textwidth}
          \centering
        \begin{tabular}{ccc}
        Computation    & Operations cost & Memory footprint \\ \hline
        Distillation basis\footnote{Generate colorvector matrix elements}& $N^3Tn^3$         & $N^3nT$      \\
        Meson elementals\footnote{Contract two matrices $\to$ tensor} & $N^3Tn^3$      & $N^3n + n^3$  \\
        Perambulators\footnote{Projection of the inverse Dirac operator $\to$ square matrices} & $N^3Tn$   & $N^3Tn$            \\
        Contractions\footnote{Contract together matrix elements and perambulators}   & $n^4T$    & $n^{3}T$   
        \end{tabular}
        \end{minipage}
        \caption{computational cost with reference time per gauge configuration and time-slice source to calculate a two point correlation function. $N$: lattice size, $n$: rank of distillation basis, $T$: lattice temporal extent}
      \end{table}

  \section{Contractions}
  We must ``tie" together the perambulators and elementals with some Dirac gamma structure, where we isolate the channel of interest ($J^{PC}$ continuum quantum numbers) with the lattice group representation. For a single meson correlator, we begin with taking the trace over square matrices, where $S$ is the distillation operator $V(t)JV^{\dagger}(t)$ \cite{Neuendorf:2024ekv} that acts on the quark fields
  \\ 

  \begin{tikzpicture}[node distance=0.1cm]  % Minimal space between nodes
    \node (box1) [draw, fill=blue!20, minimum width=1.5cm, minimum height=1.5cm] {S};
    \node (box2) [draw, fill=red!20, minimum width=1.5cm, minimum height=1.5cm, right=of box1] {$\gamma_5$};
    \node (box3) [draw, fill=blue!20, minimum width=1.5cm, minimum height=1.5cm, right=of box2] {S};
    \node (box4) [draw, fill=red!20, minimum width=1.5cm, minimum height=1.5cm, right=of box3] {$\gamma_5$};
    
    % Draw diagonal slash through box 2
    \draw (box2.north west) -- (box2.south east);
    
    % Draw diagonal slash through box 4
    \draw (box4.north west) -- (box4.south east);
    
    % Draw the "tr" label
    \node at ($(box1.west) + (-0.8,0)$) {\textbf{tr}};
    
    % Draw large brackets around the first and last boxes
    \draw[thick] 
        ($(box1.north west) + (-0.4, 0.2)$) -- ($(box1.south west) + (-0.4, -0.2)$); % left bracket
    \draw[thick] 
        ($(box4.north east) + (0.4, 0.2)$) -- ($(box4.south east) + (0.4, -0.2)$);  % right bracket
  \end{tikzpicture}
  \\ 

\begin{tikzpicture}[node distance=0.06cm]
  
  % Part 2: Sequence of shapes (rectangles and squares), colored and skinnier
  \node (vrect1) [draw, fill=green!20, minimum width=0.3cm, minimum height=2cm, below=1.5cm of box1, anchor=west] {$V$};
  \node (hrect1) [draw, fill=green!20, minimum width=2cm, minimum height=0.5cm, right=of vrect1] {$V^\dagger$};
  \node (square1) [draw, fill=blue!20, minimum width=1.5cm, minimum height=1.5cm, right=of hrect1] {S};
  
  \node (vrect2) [draw, fill=green!20, minimum width=0.3cm, minimum height=2cm, right=of square1] {$V$};
  \node (hrect2) [draw, fill=green!20, minimum width=2cm, minimum height=0.5cm, right=of vrect2] {$V^\dagger$};
  \node (square2) [draw, fill=red!20, minimum width=1.5cm, minimum height=1.5cm, right=of hrect2] {$\gamma_5$};
  
  \node (vrect3) [draw, fill=green!20, minimum width=0.3cm, minimum height=2cm, right=of square2] {$V$};
  \node (hrect3) [draw, fill=green!20, minimum width=2cm, minimum height=0.5cm, right=of vrect3] {$V^\dagger$};
  \node (square3) [draw, fill=blue!20, minimum width=1.5cm, minimum height=1.5cm, right=of hrect3] {S};

  \node (vrect4) [draw, fill=green!20, minimum width=0.3cm, minimum height=2cm, right=of square3] {$V$};
  \node (hrect4) [draw, fill=green!20, minimum width=2cm, minimum height=0.5cm, right=of vrect4] {$V^\dagger$};
  \node (square4) [draw, fill=red!20, minimum width=1.5cm, minimum height=1.5cm, right=of hrect4] {$\gamma_5$};
  % Draw diagonal slash through box 2
  \draw (square2.north west) -- (square2.south east);
    
  % Draw diagonal slash through box 4
  \draw (square4.north west) -- (square4.south east);

  % Draw large brackets around the first and last boxes
  % Draw the "tr" label
  \node at ($(vrect1.west) + (-0.8,0)$) {\textbf{tr}};
  \draw[thick] 
  ($(vrect1.north west) + (-0.4, 0.2)$) -- ($(vrect1.south west) + (-0.4, -0.2)$); % left bracket
  \draw[thick] 
  ($(square4.north east) + (0.2, 0.4)$) -- ($(square4.south east) + (0.2, -0.4)$);  % right bracket
  
\end{tikzpicture}
\\
\begin{tikzpicture}[node distance=0.1cm]  % Minimal space between nodes

  \node (box1) [draw, fill=blue!20, minimum width=0.5cm, minimum height=0.5cm] {$\tau$};
  \node (box2) [draw, fill=red!20, minimum width=0.5cm, minimum height=0.5cm, right=of box1] {$\phi_0$};
  \node (box3) [draw, fill=blue!20, minimum width=0.5cm, minimum height=0.5cm, right=of box2] {$\tau$};
  \node (box4) [draw, fill=red!20, minimum width=0.5cm, minimum height=0.5cm, right=of box3] {$\phi_t$};
  
  \draw (box2.north west) -- (box2.south east);
  
  \draw (box4.north west) -- (box4.south east);
  
  \node at ($(box1.west) + (-0.8,0)$) {= \textbf{tr}};
  
  \draw[thick] 
      ($(box1.north west) + (-0.4, 0.2)$) -- ($(box1.south west) + (-0.4, -0.2)$); % left bracket
  \draw[thick] 
      ($(box4.north east) + (0.4, 0.2)$) -- ($(box4.south east) + (0.4, -0.2)$);  % right bracket
\end{tikzpicture}
\\
\noindent
= \textbf{tr} 
\begin{tikzpicture}[baseline={(current bounding box.center)}, node distance=0.9cm]

  % Define smaller boxes without internal labels
  \node (box0) [draw, fill=blue!20, minimum width=0.4cm, minimum height=0.4cm] {};
  \node (box_q0) [draw, fill=red!20, minimum width=0.4cm, minimum height=0.4cm, right=of box0] {};
  \node (box1) [draw, fill=blue!20, minimum width=0.4cm, minimum height=0.4cm, right=of box_q0] {};
  \node (box_q1) [draw, fill=red!20, minimum width=0.4cm, minimum height=0.4cm, right=of box1] {};
  \node (box2) [draw, fill=blue!20, minimum width=0.4cm, minimum height=0.4cm, right=of box_q1] {};
  \node (box_q2) [draw, fill=red!20, minimum width=0.4cm, minimum height=0.4cm, right=of box2] {};
  \node (box3) [draw, fill=blue!20, minimum width=0.4cm, minimum height=0.4cm, right=of box_q2] {};
  \node (box_q3) [draw, fill=red!20, minimum width=0.4cm, minimum height=0.4cm, right=of box3] {};
  
  % Attach the subscript and time labels close to the boxes with better spacing
  \node at ($(box0.east) + (0.15, 0)$) {\small $_0(0)$};
  \node at ($(box_q0.east) + (0.26, 0)$) {\small $_{q0}$(t$_0$, t$_1$)};
  \node at ($(box1.east) + (0.22, 0)$) {\small $_1$(t$_1$)};
  \node at ($(box_q1.east) + (0.26, 0)$) {\small $_{q1}$(t$_1$, t$_2$)};
  \node at ($(box2.east) + (0.21, 0)$) {\small $_2$(t$_2$)};
  \node at ($(box_q2.east) + (0.26, 0)$) {\small $_{q2}$(t$_2$, t$_3$)};
  \node at ($(box3.east) + (0.21, 0)$) {\small $_3$(t$_3$)};
  \node at ($(box_q3.east) + (0.26, 0)$) {\small $_{q3}$(t$_3$, t$_0$)};
  
  % Draw large brackets around the first and last boxes, ensuring they're outside all elements
  \draw[thick] 
      ($(box0.north west) + (-0.6, 0.2)$) -- ($(box0.south west) + (-0.6, -0.2)$); % left bracket
  \draw[thick] 
      ($(box_q3.north east) + (1.1, 0.2)$) -- ($(box_q3.south east) + (1.1, -0.2)$);  % right bracket

\end{tikzpicture}


where the colored boxes in the last diagram are $4N_V \times 4N_V$ matrices and $t,q$ go up to $N-1$. A contraction must be carrried out for every gauge configuration (we use 200 in this study); On each configuration, every timeslice and time source is accessed.  

% \todo{correct this for single node jobs}
% \begin{table}[!h]
%   \centering \footnotesize
%   \hspace*{-0.6cm}
%   \begin{tabular}{|c|c|c|c|c|c|c|c|c|c|c|c|}
%   \hline
%   \multirow{2}{*}{$\beta$}  & \multirow{2}{*}{$m_{ud}$}   & \multirow{2}{*}{$m_{s}$}   & \multirow{2}{*}{$L^3 \times T$}  & \multirow{2}{*}{$N_\mathrm{cnfg}$} & \multicolumn{3}{c|}{Eigenvectors} & \multicolumn{4}{c|}{Perambulators} \\
%   & & & & & Time [s] & Nodes & Cost [kch] & Time [s] & Nodes & $N_\mathrm{srcs}$ & Cost [kch] \\
%   \hline \hline
%   $3.3$    & $-0.1233$  & $-0.057$  & $24^3\times64$  & 500  & 1190 & 1 & 21  & 42  & 2 & 8 & 11  \\ \hline
%   $3.7$    & $-0.0200$  & $-0.0$    & $32^3\times96$  & 200  & 4230 & 1 & 30  & 151 & 2 & 8 & 17  \\ 
%   $3.7$    & $-0.0220$  & $-0.0$    & $32^3\times96$  & 200  & 4230 & 1 & 30  & 151 & 2 & 8 & 17  \\ 
%   $3.7$    & $-0.0250$  & $-0.0$    & $40^3\times96$  & 200  & 8260 & 1 & 48  & 295 & 2 & 8 & 33  \\ \hline
%   $3.57$   & $-0.0380$  & $-0.007$  & $24^3\times64$  & 200  & 1190 & 1 & 8   & 42  & 2 & 8 & 4   \\ 
%   $3.57$   & $-0.0440$  & $-0.007$  & $32^3\times64$  & 200  & 2820 & 1 & 20  & 100 & 2 & 8 & 11  \\ 
%   $3.57$   & $-0.0483$  & $-0.007$  & $48^3\times64$  & 200  & 9520 & 1 & 67  & 340 & 2 & 8 & 38  \\ \hline
%   $3.3$    & $-0.1200$  & $-0.057$  & $16^3\times64$  & 500  & 360  & 1 & 5   & 12  & 2 & 8 & 3   \\ 
%   $3.3$    & $-0.1265$  & $-0.057$  & $24^3\times64$  & 500  & 1190 & 1 & 21  & 42  & 2 & 8 & 11  \\ \hline
%   \end{tabular}
%   \caption{Computer time requirements for measurements on the ensembles. The first row is based on the test runs we have made, while the following ones are based on the volume scaling of the problem.}
%   \label{tab:ensembles}
%   \end{table}



\section{Ensemble Details}
We generated ensembles with $N_f = 2+1$ quark flavors, with 6-stout dynamical smeared Wilson fermions and a tree-level Symanzik improved gluon action. Having several $a$ and $m_{\pi}$ at our disposal will permit a systematic study of $m_{\pi}$ dependence of doubly charmed exotics. Since the doubly charmed tetraquark of interest is close to $D^0D^0\pi^+$ threshold, it is sensitive to $m_{ud}$, thus, it is advantageous to have many ensembles. 
We are using the action and parameters already employed and tuned for Ref. \cite{Durr:2008zz}. We used the ensemble indicated by * to perform this preliminary study, informing the ideal distillation parameters for the remaining ensembles to optimize computational resources.

\begin{table}[!h]
  \centering \footnotesize
  \begin{tabular}{|c|c|c|c|c|c|c|}
  \hline
  $\beta$  & $m_{ud}$   & $m_{s}$   & $L^3 \times T$  & $a$ [fm]   & $m_\pi$ [MeV]   & \# of traj. \\ \hline \hline
  $3.7$    & $-0.0200$  & $-0.0$    & $32^3\times96$  & $0.065$      & $420$         &  2000      \\ 
  $3.7*$    & $-0.0220$  & $-0.0$    & $32^3\times96$  & $0.065$      & $380$         &  2000      \\ 
  $3.7$    & $-0.0250$  & $-0.0$    & $40^3\times96$  & $0.065$      & $300$         &  2000      \\ \hline
  $3.57$   & $-0.0380$  & $-0.007$  & $24^3\times64$  & $0.085$      & $420$         &  2000      \\ 
  $3.57$   & $-0.0440$  & $-0.007$  & $32^3\times64$  & $0.085$      & $300$         &  2000      \\ 
  $3.57$   & $-0.0483$  & $-0.007$  & $48^3\times64$  & $0.085$      & $200$         &  2000      \\ \hline
  $3.3$    & $-0.1200$  & $-0.057$  & $16^3\times64$  & $0.125$      & $400$         &  5000      \\ 
  $3.3$    & $-0.1233$  & $-0.057$  & $24^3\times64$  & $0.125$      & $330$         &  5000      \\ 
  $3.3$    & $-0.1265$  & $-0.057$  & $24^3\times64$  & $0.125$      & $280$         &  5000      \\ \hline
  \end{tabular}
  \caption{Physical parameters of the ensembles that we have available for this project. Bare parameters have been taken from~\cite{Durr:2008zz}.}
  \end{table}


