% !TEX root = mythesis.tex

%==============================================================================
\chapter{Computational Setup}
\label{sec:comp}
%==============================================================================

Distillation is costly initially both in storage and component construction. This pays off at the end of the day as we can reuse the perambulators for subsequent calculations of different systems. For the di-meson system we are investigating, the contraction cost is not the dominant contribution. We will use the MultiGrid (MG) solver from \texttt{QUDA}, \texttt{Chroma} with \texttt{Superbblas} support, the \texttt{PRIMME} eigensolver, and \texttt{Numpy Einsum} for contractions. The amount of computation and storage scales with the lattice size $N$ and the rank of the distillation basis, $n$. The optimal rank of the distillation basis is determined experimentally, but it is proportional to the spatial volume of the lattice. All of the high-peformance computing tasks below were performed on the Jureca cluster at the Juelich Supercomputing Centre. 
\section{Pipeline}
\begin{figure}[!htbp]
    \begin{center}
      \scalebox{1.0}{
        \begin{tikzpicture}[node distance=1.5cm and 3cm] % Adjusted node distance
            \tikzstyle{startstop} = [rectangle, rounded corners, minimum width=3cm, minimum height=1cm, text centered, draw=black, fill=red!30]
            \tikzstyle{io} = [trapezium, trapezium left angle=70, trapezium right angle=110, minimum width=3cm, minimum height=1cm, text centered, draw=black, fill=blue!30]
            \tikzstyle{process} = [rectangle, minimum width=3.5cm, minimum height=1cm, text centered, text width=3cm, draw=black, fill=orange!30]
            \tikzstyle{arrow} = [thick,->,>=stealth]

            % Nodes
            \node (start) [startstop] {Input File Generation: XML and SLURM Batch Scripts};
            \node (in1) [io, below of=start] {Calculate Distillation Basis/Eigenvectors};
            
            % Parallel processes
            \node (pro1) [process, below of= in1, xshift=-3cm, yshift=-1cm] {Meson Elementals};
            \node (pro2) [process, below of= in1, xshift=cm, yshift=-1cm] {Perambulators};
            
            % H5 Conversion processes
            \node (h5conv1) [process, below of=pro1] {H5 Conversion};
            \node (h5conv2) [process, below of=pro2] {H5 Conversion};
            
            % Contracting
            \node (contract) [process, below of=h5conv1, xshift=3cm] {Contract Elementals and Perambulators};
            
            % Output
            \node (out1) [io, below of=contract] {Output the Two-Point Correlator};
            \node (stop) [startstop, below of=out1] {H5 output ($N_t \times N_{cfg}$)};

            % Arrows
            \draw [arrow] (start) -- (in1);
            \draw [arrow] (in1) -- (pro1);
            \draw [arrow] (in1) -- (pro2);
            \draw [arrow] (pro1) -- (h5conv1);
            \draw [arrow] (pro2) -- (h5conv2);
            \draw [arrow] (h5conv1) -- (contract);
            \draw [arrow] (h5conv2) -- (contract);
            \draw [arrow] (contract) -- (out1);
            \draw [arrow] (out1) -- (stop);
        \end{tikzpicture}
      }
    \end{center}
    \caption{Flowchart of the computational workflow for computing lattice objects with distillation.}
    \label{fig:alg_flow}
\end{figure}

\subsection{Eigenpairs/Distillation Basis Calculation}
This is the first step in the calculation which ``seeds'' the subsequent calculation of elementals and perambulators. We use the \verb|Primme| eigensolver \todo{cite this}  

\subsection{Elementals Calculation}

\subsection{Perambulators Calculation}

\section{Cost and Storage of Distillation}
% \cite{romero_efficient_2020}.
      \vspace{1em}
       
       % The product of        %  \texttt{max_tslices_in_contraction} \texttt{max_moms_in_contraction} and \texttt{max_vecs} controls how much memory is used for contractions. Their tuning is more critical when using GPUs. The optimal value is usually the largest values of the parameters that the CPU or device can handle.
        % \begin{table}
         \begin{minipage}{16cm}
        \hspace*{2em}\begin{tabular}{ccc}
        Computation    & Operations cost & Memory footprint \\ \hline
        Distillation basis\footnote{Generate colorvector matrix elements}& $N^3Tn^3D$         & $N^3nT$      \\
        Meson elementals\footnote{Contract two matrices $\to$ tensor} & $N^3Tn^3$      & $N^3n + n^3$  \\
        Perambulators\footnote{Projection of the inverse Dirac operator $\to$ square matrices} & $N^3Tn$   & $N^3Tn$            \\
        Contractions\footnote{Contract together matrix elements and perambulators}   & $n^4T$    & $n^{3}T$   
        \end{tabular}
        \end{minipage}
        once a suitable set of perambulators compute, \textbf{reuse} to correlate a collection of interpolators
        
In order to perform spectroscopy calculations for a given ensemble, there exists a sequential dependency chain:
\begin{itemize}
     \setlength\itemsep{1em}
        \item[\checkmark] \textbf{Ensemble generation:} $N_f = 2+1$ quark flavors, a tree level Symanzik improved gluon action and 6-stout dynamical smeared Wilson fermions
        \item[\checkmark] \textbf{HPC Tasks:} Generation of distillation basis, perambulators, meson elementals using \texttt{Chroma} with \texttt{superbblas} support on the Jureca cluster at JSC
        \item Construct \textbf{Di-meson distilled operators} using Hadspec method of subduction coefficents and helicity operators
        \item[\checkmark] Perform \textbf{contractions} of multi-hadron operators $\to$ 2pt correlators
        \item Construct correlation function coming from the \textbf{GEVP} in the right irreducible representation
        \item \textbf{Compute spectrum} and energy shifts w.r.t to the $DD^*$ threshold for a heavy quark mass close to the charm quark mass.
        \item \textbf{L\"uscher analysis} to obtain finite volume energies from Scattering amplitudes
        \item \textbf{Search for Poles} AKA when an attractive potential is not deep enough to hold a bound state
    \end{itemize}

\section{Ensemble Details}
We generated ensembles with $N_f = 2+1$ quark flavors, with 6-stout dynamical smeared Wilson fermions and a tree-level Symanzik improved gluon action. Having several $a$ and $m_{\pi}$ at our disposal permit a systematic study of $m_{\pi}$ dependence. Since the doubly charmed tetraquark of interest is close to $D^0D^0\pi^+$ threshold, it is sensitive to $m_{ud}$, thus, it is advantageous to have many ensembles. 
\\ 

\begin{tabular}{ccccccc}
\multirow{5}{*}{\substack{$\beta = 3.30$\\ \quad$a = 0.125[fm]$ }} & & $m_{ud}$ & $m_{s}$ & $L^3 \times T$ & $m_\pi$ [MeV] & $N_{conf}$\\
\cmidrule{3-5} \cmidrule{6-7}
& & $-0.1309$ & $-0.057$ & $48^3\times64$ & $135$ & * \\
& & $-0.1291$ & $-0.057$ & $32^3\times64$ & $200$ & * \\
& & $-0.1265$ & $-0.057$ & $24^3\times64$ & $280$ & 1000 \\
& & $-0.1233$ & $-0.057$ & $24^3\times64$ & $330$ & 1000 \\
& & $-0.1200$ & $-0.057$ & $16^3\times64$ & $400$ & 1000 \\
\midrule
\multirow{5}{*}{\substack{$\beta = 3.57$\\ \quad {$a = 0.085[fm]$} }} && $m_{ud}$ & $m_{s}$ & $L^3 \times T$ & $m_\pi$ [MeV] & $N_{conf}$\\
\cmidrule{3-5} \cmidrule{6-7}
& & $-0.0498$ & $-0.007$ & $64^3\times96$ & $135$ & * \\
& & $-0.0483$ & $-0.007$ & $48^3\times64$ & $200$ & 400 \\
& & $-0.0440$ & $-0.007$ & $32^3\times64$ & $300$ & 400 \\
& & $-0.0380$ & $-0.007$ & $24^3\times64$ & $420$ & 400 \\
\midrule
\multirow{5}{*}{\substack{$\beta = 3.70$ \\ \quad $a = 0.065[fm]$}} && $m_{ud}$ & $m_{s}$ & $L^3 \times T$ & $m_\pi$ [MeV] & $N_{conf}$\\
\cmidrule{3-5} \cmidrule{6-7}
& & $-0.02981$ & $-0.0$ & $64^3\times96$ & $135$ & * \\
& & $-0.02855$ & $-0.0$ & $64^3\times96$ & $200$ & * \\
& & $-0.0250$ & $-0.0$ & $40^3\times96$ & $300$ & 400 \\
& & $-0.0220$ & $-0.0$ & $32^3\times96$ & $380$ & 400 \\
& & $-0.0200$ & $-0.0$ & $32^3\times96$ & $420$ & 400 \\
\end{tabular}



\section{Two-Point Correlation function} 